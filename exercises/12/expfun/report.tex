\documentclass{article}
\usepackage{amsmath}
\usepackage{graphicx}
\usepackage[utf8]{inputenc}
\usepackage{kpfonts}
\usepackage[T1]{fontenc}

\begin{document}
\title{Exponential function via ODE}
\author{Christian Færgemand Madsen}
\date{}
\maketitle

\noindent The exponential function can be implemented by solving the following ODE:
\begin{equation}
\label{eq:expODE}
\frac{\mathrm{d} y}{\mathrm{d} x}=y(x)
\end{equation}
subject to the initial condition: $y(0)=1$. In order to avoid numerical integrations over large intervals the argument have been reduced such that $0 \leq x <1$ by using the following equations:
\begin{equation*}
\text{exp}(-x)=\frac{1}{\text{exp}(x)}, \ \text{exp}(x)=\left [ \text{exp}\left ( \frac{x}{2} \right ) \right ]^2
\end{equation*}
This differential equation has been solved numerically by using the GNU Scientific Library. The used algorithm is the Runge-Kutta-Fehlberg (4,5) method. The numerical solution is within a relative error of $10^{-10}$ of the correct result. A comparison between the exponential function found in the library \texttt{<math.h>} and the one obtained by solving \eqref{eq:expODE} can be seen in figure \ref{fig:expODEplot}.

\begin{figure}[h]
\centering
\input{plot.tex}
\caption{Comparison between the exponential function from the library math.h and the exponential function obtained by solving \eqref{eq:expODE}.}
\label{fig:expODEplot}
\end{figure}
\end{document}